%----------------------------------------------------------------------------------------
%	PACKAGES AND OTHER DOCUMENT CONFIGURATIONS
%----------------------------------------------------------------------------------------

\documentclass[12pt]{report} 
% Default font size is 12pt, it can be changed here

\usepackage{appendix}
\usepackage{geometry} % Required to change the page size to A4
\geometry{a4paper} % Set the page size to be A4 as opposed to the default US Letter

\usepackage{graphicx} % Required for including pictures

\usepackage{float} % Allows putting an [H] in \begin{figure} to specify the exact location of the figure
\usepackage{wrapfig} % Allows in-line images such as the example fish picture

\usepackage{lipsum} % Used for inserting dummy 'Lorem ipsum' text into the template

\linespread{1.2} % Line spacing

%\setlength\parindent{0pt} % Uncomment to remove all indentation from paragraphs

\graphicspath{{Pictures/}} % Specifies the directory where pictures are stored

\begin{document}

%----------------------------------------------------------------------------------------
%	TITLE PAGE
%----------------------------------------------------------------------------------------

\begin{titlepage}

\newcommand{\HRule}{\rule{\linewidth}{0.5mm}} % Defines a new command for the horizontal lines, change thickness here

\center % Center everything on the page

\textsc{\LARGE Vrije Universiteit Brussel}\\[1.5cm] 
\textsc{\Large Operating Systems and Security}\\[0.5cm] 
\textsc{\large Mini-project essay}\\[0.5cm] 

\HRule \\[0.4cm]
{ \huge \bfseries Title}\\[0.4cm] % Title of your document
\HRule \\[1.5cm]

\begin{minipage}{0.4\textwidth}
\begin{flushleft} \large
\emph{Author:}\\
Pieter \textsc{Libin} % Your name
\end{flushleft}
\end{minipage}
~
\begin{minipage}{0.4\textwidth}
\begin{flushright} \large
\emph{Professor:} \\
Prof. Dr. Martin \textsc{Timmerman} 
% Supervisor's Name
\end{flushright}
\end{minipage}\\[4cm]

{\large \today}\\[3cm] % Date, change the \today to a set date if you want to be precise

%\includegraphics{Logo}\\[1cm] % Include a department/university logo - this will require the graphicx package

\vfill % Fill the rest of the page with whitespace

\end{titlepage}

%----------------------------------------------------------------------------------------
%	TABLE OF CONTENTS
%----------------------------------------------------------------------------------------

\tableofcontents % Include a table of contents

\newpage % Begins the essay on a new page instead of on the same page as the table of contents 

\chapter{Introduction} % Major section

\chapter{General scope of availability and scalability}

\chapter{Different aspects in cluster availability}

\chapter{Scalability and its effects on economy and environment}

\chapter{Identifying high availability and scalability solutions}
\section{Cluster nodes}
\subsection{Cluster node monitoring}
The two major monitoring tools on GNU/Linux platforms are Nagios and
Ganglia. Both tools implement several monitoring aspects, but the
focus of each of the tools is different. Nagios is more oriented
towards detecting system problems and sending out notificiations for
such events, while Ganglia provides a long term overview of the status
of cluster nodes. To ensure high availability, detecting problems is
the most important aspect. The sooner problems are revealed, the
sooner actions can be undertaken to avoid any impact on end users. It
is imporant that problems are properly communicated to system
administrators, since they might be able to further investigate the
problem. In the event of any hardware failure the intervention of
an administrator will be required. However, it is also necessary that
 it is possible to execute programs repair the state of the cluster
 upon the detection of failure. Nagios specializes on both these
 issues \cite{nagios:2013}.
In order to make new nodes available when necessary, we need to
monitor the resource availability on all of the online nodes.
Monitoring statistics also allows system engineers to gain insight in
performance needs of an application over the course of time. 
Such statistics can also be used to setup models that can be used to
predict system load \cite{andreolini:2006}. Ganglia collects these
kind of statistics and stores them in a round-robin database, namely RDDTool
\cite{ganglia:2013} \cite{rrdt:2013}.
Nagios is free software (GPL) and Ganglia's source code is licensed
under the BSD license. Note that there also exists a commercial
version of the Nagios software: Nagios XI. Nagios XI extends the free
software edition by providing a richer and more powerfull web interface.

\section{Computational cluster nodes}
\subsection{Application monitoring}
Monitoring a cluster node is not enough to garantee that it is
properly servicing clients. It is possible that the application (for
example: the web application server) has gotten in an infinite loop,
and as such is using practically all CPU, but in reality is
processing no new client request.
A possibility to monitor applications is to keep track of the number of
client request per seconds (or another appropriate time unit) that get
processed and compare it with the amount of computational resources
that is used.
This number of client requests is very application specific; a web
server might process thousands client requests per second, while a
server responsable for rendering a scene in an animated movie might only process
one request per hour \cite{apm:2013}.
This information usually can be extraced from the log files generated
by the server application (for example: the request log file for
Apache httpd). This information than can be pass to Ganglia to be
stored and monitored together with the other server's statistics
\cite{ganglia:2013}.

\subsection{Cluster node management}
In large cluster environments, it is not possible to wait for
a human intervention when a problem occurs. Therefore, when nodes or
applications fall away, the monitor server should restart them.
There are several systems that allow for programs to restart
physical servers that reside in a rack (for example Dell's DRAC).
Since I will only use a simulated cluster in this mini-project, I will
only deal with starting and stopping virtualized nodes (which I think should
be a valuable contribution taken into account the large amount of
virtualized server that are used in datacenters these days).
I will use VirtualBox to set up my simulated cluster, this
virtualization software has support to start and stop virtual machines
using the command line.
When an application stalls but the operating system is still fully
functioning, the monitoring server can decide to restart the server application, or
(if the application has support for this) simply kill the thread that
is responsible for the problems.

\subsection{Load balancing proxies}
Load balancing is a method for distributing workloads across multiple
computing nodes. It can be used to improve the reliabilty of a system through
redundancy and to provide scalability by sending requests the most
available node.
Load balancing has several applications such as High Performance
Clusters, database servers, and HTTP servers. In this section I will
focus on HTTP load balancing.
A prominent TCP/HTTP load balancing proxy is HAProxy.
\cite{haproxy:2013}. HAProxy is free software (GPLv2 for most of the
code, LGPLv2 for some of the headers). This software supports a set of
balancing algorithms \cite{tarreau:2006}, most notably:
\begin{itemize}
  \item round robin: each server is used in turns, taking into account
    the different server's weights (these weights can be configured
    per server and can be adjusted on the fly)
  \item balance leastconn: the server with the lowest number of connections receives the connection
  \item source/uri: this algorithm hashes respectively the source IP
    or URI of the request, this hash is than divided by the total
    weight of the running servers to determine which server should
    receive the request
\end{itemize}
HAProxy also supports Access Control Lists configurations that allow
the load balancer to select a server based on a header in the HTTP
request. When multiple server clusters exists in different countries
and/or continents, this feature allows us to connect a user to a
server that is located near his own location.
Another important aspect to consider when setting up a load balancing
infrastructure for HTTP servers is that many dynamic web applications
rely on session context \cite{tarreau:2006}. Since this session context is usually stored
on the web server, it is necessary for the load balancer to pass the
client's request to the same web server: this is called persistence.
HAProxy provides session context persistence through `cookie
learning' or `cookie insertion'. When using `Cookie learning' the load
balancer will store a connection between an appliction cookie (this
application cookie has to be configured) and a server. The
disadvantages of this approach are:
\begin{itemize}
  \item the application cookie -> server mapping requires memory
  \item if the load balancer crashes, there is no way to connect the
    client back to the server
\end{itemize}
`Cookie insertion' avoids these problems by adding a cookie that
identifies the server to the response that is sent to the client. By
using this approach the load balancer can simply reuse the
cookie value presented by the user to pass the request to the correct
server.

\section{Storage}

\subsection{Data replication}
Data replication implies that the same date is stored on multiple
data storage systems. This should be transparant to the end user: the
user should not be aware of this when using the data (note that when
working with a multi-tier architecture, it is possible that the
`end-user'  is represented by another tier in the system).
Data replication is relevant for both:
\begin{itemize}
\item high availability systems: to ensure a smooth transition from a
  failing data storage node to a working node
\item systems that require scalability: to make load balancing possible
\end{itemize}
We can distinguish 3 types of data replication
\cite{datareplication:2013} :

\subsubsection{Database replication}

\paragraph*{Master-slave replication}
In this setup, there exists one master database, this database is the only instance
that accepts updates. When an update statement is received by the
master database, it is applied to the database and appended to the log. The statements
in this log are than propagated to the slave databases.
Most database systems support this replication strategy (Postgres \cite{postgres_db:2013},
MySQL \cite{mysql_db:2013}, Oracle \cite{oracle_db:2013}, ...).

\paragraph*{Multi-master replication}
In this setup, multiple master instances exists, each of these master
instances accept updates. These updates are than communicated with the
other master instances. 
This has the advantages that :
\begin{itemize}
\item the master node can fail without interrupting the system
\item the update load can be distributed over multiple nodes
\end{itemize}
There are also some significant disadvantages related to this technique:
\begin{itemize} 
\item increased complexity
\item most implementations violate the ACID constraints
\item to fix conflicts that may arise between different database
  instances, resources of the database nodes are required and
  communication between the conflicted nodes will increase the network
  traffic
\end{itemize}
Note that there are other techniques to accomodate the advantage of
load balancing mentioned above: database shards, NoSQL (reference to
this section).

\chapter{Creating a `Wikpedia clone' cluster}
interesting hints to improve performance: Tarreau:2006 article: p
17-18 (printed)

\chapter{Conclusion} % Major section
\section{What is lacking}
The ability to replicate journaled filesystems (by using the journal),
like in MS DPM.

\begin{appendices}
\chapter{Concepts}
Round robin (refer to this appendix from the places where round robin
was used)
ACID
\end{appendices}

%----------------------------------------------------------------------------------------
%	BIBLIOGRAPHY
%----------------------------------------------------------------------------------------



\bibliography{mybib}{}
\bibliographystyle{ieeetr}

%----------------------------------------------------------------------------------------

\end{document}
